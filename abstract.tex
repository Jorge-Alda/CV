\documentclass[combined.tex]{subfiles}
%\usepackage[margin=2cm]{geometry}

\begin{document}
In this talk I will present the ``ALP Automatic Computing Algorithm (ALPaca)'', a Python package that allows to study the phenomenology of Axion-like particles (ALPs) in collider experiments in a completely general framework. ALPaca implements matching to several UV models, running the ALP couplings under the Renormalization Group equations, calculating the ALP production cross sections and decay widths, and confronting the results with a large library of experimental data. As an example, I will show the results of the application of ALPaca to the study of the anomaly in $B^+\to K^+ \nu \overline{\nu}$ decays found by Belle II, interpreting it as a $B^+\to K^+ a$ process where the ALP is a light, long-lived particle that decays outside the detector and/or to a dark sector.
\end{document}