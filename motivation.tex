\documentclass[12pt, a4paper]{article}
\usepackage[margin=2.5cm]{geometry}

\begin{document}
Since I was a little kid, I have always been fascinated with the patterns and regularities that govern the world around us. That fascination motivated me to study Physics, and to focus on the theoretical aspects regarding the fundamental particles and forces.

{}~

While the current Standard Model has achieved an incredible degree of precision at describing our world, there are still many open questions awaiting for an answer, motivating the work of theoretical physicist. The aspects of physics beyond the Standard Model more interesting to me are the following:
\begin{itemize}
\item The existence of three generations of particles, with identical fundamental properties other than their interactions with the Higgs field, and therefore, their masses, is a great mystery. Even more intriguing is the wide disparity in their masses, ranging from the top quark, roughly the mass of a gold atom and a Yukawa coupling of order one, down to the electron, with a mass six orders of magnitud smaller.
\item Of course, we cannot forget the elusive neutrinos, which could be Dirac or Majorana particles, and whose mass is possibly six orders of magnitude smaller than that of the electron.
\item And finally, the strong CP problem concerning the quantum chromodynamics. Many proposals to fix this problem include a new global $U(1)$ symmetry which is spontaneously broken. The pseudo-Goldstone boson associated to this symmetry, an axion or axion-like particle, should be a light pseudo-scalar particle, leaving its imprint in astrophysics, cosmology and colliders.
\end{itemize}
  In my opinion, there must be some kind of pattern hidden in the flavour sector linking them all, and I would love to be part of the team that solves this mystery.

{}~


In the recent years, LHC and the $B$ factories BaBar and Belle have found some interesting results in the $R_K$ and $R_D$ ratios that measure possible deviations from Lepton Flavour Universality, a clear prediction of the Standard Model. If these anomalies turn out to be real, it would be necessary to look at related processes, such as $B\to K^{(*)} \bar{\nu}\nu$, that could show similar deviations. Interesting times are ahead, and it might be the key to the flavour realm!

{}~

The pathway to examine these anomalies is twofold: First we need a model-independent approach, such as Effective Field Theories, to diagnose the departures from the Standard Model. And second, we also need some model framework, such as axion-like particles coupling to fermions, to focus the search of the New Physics responsible of these anomalies. In my time as PhD student I have combined both approaches.

{}~

On the technical side, New Physics coupling at a certain energy scale $\Lambda$ can manifest at many different experiments in lower scales, due to the mixing generated by the Renormalization Group equations of the dimension-6 effective operators. For this reason, it is convenient to check the consistency of the predictions of any theoretical model against an ever-growing list of experimental observations. The task at hand requires the use of computer codes and, in recent years, also Machine Learning techniques.

{}~

I envision my work as a postdoctoral researcher as an opportunity to investigate the flavour structure of the Standard Model, including the $B$-anomalies, neutrinos and axion-like particles, and to employ the state-of-the-art analytical and computational methods to do so. I sincerely look forward to working and collaborating with you in these topics!

\end{document}